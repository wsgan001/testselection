\section{Background}
\label{sec:background}

This section describe background of XACML policies, policy models, and model checking.


\begin{figure}[t]%{t}
%\begin{figure}[firstnumber=100]
\begin{CodeOut}
\begin{alltt}
 1 <Policy PolicyId="\textbf{Library Policy}" RuleCombAlgId="\textbf{Permit-overrides}">
 2  <Target/>
 3    <Rule RuleId="\textbf{1}" Effect="\textbf{Permit}">
 4      <Target>
 5        <Subjects><Subject> \textbf{BORROWER} </Subject></Subjects>
 6        <Resources><Resource> \textbf{BOOK} </Resource></Resources>
 7        <Actions><Action> \textbf{BORROWERACTIVITY} </Action></Actions>
 8      </Target>
 9	    <Condition>
10        <AttributeValue> \textbf{WORKINGDAYS} </AttributeValue>
11      </Condition>
12    </Rule>
13    <Rule RuleId="\textbf{2}" Effect="\textbf{Deny}">
14      <Target>
15        <Subjects><Subject> \textbf{BORROWER} </Subject></Subjects>
16        <Resources><Resource> \textbf{BOOK} </Resource></Resources>
17        <Actions><Action> \textbf{BORROWERACTIVITY} </Action></Actions>
18      </Target>
19	    <Condition>
20        <AttributeValue> \textbf{HOLIDAYS} </AttributeValue>
21      </Condition>
22    </Rule>
23    <Rule RuleId="\textbf{3}" Effect="\textbf{Permit}">
24      <Target>
25        <Subjects><Subject> \textbf{SECRETARY} </Subject></Subjects>
26        <Resources><Resource> \textbf{BOOK} </Resource></Resources>
27        <Actions><Action> \textbf{FIXBOOK} </Action></Actions>
28      </Target>
29	    <Condition>
30        <AttributeValue> \textbf{MAINTENANCEDAY} </AttributeValue>
31      </Condition>
32    </Rule>
33      <!-- A final, "fall-through" rule that always Denies -->
34    <Rule RuleId="\textbf{FinalRule}" Effect="\textbf{Deny}"/>
35 </policy>
\end{alltt}
\end{CodeOut}
\vspace*{-3.0ex} \caption{An example policy specified in XACML}
 \label{fig:example}
\end{figure}
\subsection{XACML Policies}

XACML (eXtensible Access Control Markup
Language)~\cite{oasis05:xacml} is a language specification standard
%designed by OASIS.
published by Organization for the Advancement of Structured
Information Standards (OASIS).
An XACML access control specification consists of a policy set and a
policy combining algorithm.


XACML supports for various policy models such as Role Based Access Control (RBAC)~\cite{anderson04:rbacxacml,ferraiolo01:proposed} and organization-based access control (OrBAC)~\cite{kalam03:orBac}. For RBAC and OrBAC, various job functions are associated to roles (e.g., staff or employee) that a user possesses. Permissions or denials to take an action on certain objects are assigned to specific roles (instead of users). OrBAC supports for 
extra features such as role, activity (i.e., action), and view (i.e., object) hierarchy. Figure~\ref{fig:example} is an OrBAC policy example since this example expresses roles such as ``Secretary'' and ``Borrower'' are associated with subject attributes and policy decisions are made through specific roles.


Let $Conditions$ be the set of the environmental context
such as working days or holidays, a file size, and a user's task authorization level. 
Let $S$, $O$, $A$, and $C$ denote all the subjects,
objects, actions, and  $Conditions$, respectively, in an access control system.

An XACML policy consists of a \Intro{policy set}, which consists
of \Intro{policy sets} and \Intro{policies}. A \Intro{policy} consists
of a sequence of \Intro{rules}, each of which
specifies under what conditions $C$ subject $S$ is allowed or denied
to perform action $A$ (e.g., read) on certain object (i.e., resources) $O$ in a system.

A rule $R$ is of the following form:
\begin{center}
$R$ $\subseteq$ $S$ $\times$ $A$ $\times$ $O$ $\times$ $C$ $\longrightarrow$ $Dec$
\end{center}
where $Dec$ denotes a decision, which is either permit or deny.
A user's request is evaluated against a policy.
A request matches with a rule if the request satisfies the rule's
attributes. Then the rule's decision can be returned.

For R1 = 
Formally, a request $Q$ in the following form matches $R$:
\begin{center}
$Q$ $\subseteq$ $S_q$ $\times$ $A_q$ $\times$ $O_q$ $\times$ $C_q$ where $S_q$ $\subseteq$ $S$, $A_q$ $\subseteq$ $A$, $O_q$ $\subseteq$ $O$, and $C_q$ $\subseteq$ $C$.
\end{center}


More than one rule in a policy may be applicable to a given request.
The \Intro{rule combining algorithm} is used to combine multiple
rule decisions into a single decision. There are four standard rule
combining algorithms: \CodeIn{deny-overrides}, \CodeIn{permit-overrides}, \CodeIn{first
applicable}, and \CodeIn{only-one-applicable}. The
\Intro{deny-overrides algorithm} returns \CodeIn{Deny} if any rule
evaluation returns \CodeIn{Deny} or no rule is applicable. The
\Intro{permit-overrides algorithm} returns \CodeIn{Permit} if any
rule evaluation returns \CodeIn{Permit}. Otherwise, the algorithm
returns \CodeIn{Deny}.
The \Intro{first-applicable algorithm} returns whatever the
evaluation of the first applicable rule returns. The
\Intro{only-one-applicable} algorithm returns the decision of the only
applicable rule if there is only one applicable rule, and returns
error otherwise.

Figure~\ref{fig:example} shows an example policy specified
in XACML. The example policy consists of three XACML rules used in our real-life library access
control policy subject, called $LSM$~\cite{mouelhi09:tranforming}.
Note that we simplified XML formats to reduce
space for this example.
Lines 3-12 describe a rule that \CodeIn{borrower} is Permitted for \CodeIn{book} \CodeIn{borroweractivity} (e.g., borrowing books) in \CodeIn{working days}.
Lines 13-22 describe
a rule that \CodeIn{borrower} is Denied for \CodeIn{book} \CodeIn{borroweractivity} in \CodeIn{holidays}.
Lines 23-32 describe
a rule that \CodeIn{secretary} is Permitted for \CodeIn{book} \CodeIn{fixing} in \CodeIn{maintenance day}.
As these rules are combined using the permit-overrides
algorithm, the \Intro{permit-overrides algorithm} returns the
\CodeIn{Permit} if a request matches at least one of
rules with permit decisions. For example, if rules 2 and 3 are applicable to a request, the decision
evaluated from rule 3 is given higher priority than that of rule 2.


\subsection{Regression Testing + PDP + PEP}






