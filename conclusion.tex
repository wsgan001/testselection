\section{Conclusion}
\label{sec:conclusion}

We make three key contributions in this paper. First, we
proposed a test selection approach for
access control policy evolution.
To the best of our knowledge,
our paper is the first one for automatic test-selection approach in context of policy
evolution.
We present three automatic test-selection techniques.
Second, we presented a test augmentation technique to
select test cases for modification to cover not-covered
policy changes. Third, we conduct
evaluation with three metrics
to measure the effectiveness of our approach
and the efficiency of our three test selection
techniques.
The evaluation results demonstrated that our approach is effective to select
test cases for test reduction.
Among the proposed three test-selection techniques, the evaluation results demonstrated that the technique based recorded
request evaluation is the most efficient compared with other
two techniques. Besides, we also achieve 100\% coverage
of changed policy behaviors with augmented test cases.


%the first
%  and the second techniques in terms of elapsed time. 


%If 
%
%generate additional new test cases to cover not-covered-impacted-rules with selected test cases by the preceding techniques.
%
%to select every test case impacted by policy changes.
%
%
%from existing test cases to test program code impacted by policy changes.
%
%comprehensive fault model for ?rewall
%policies, including ?ve types of faults. For each type of
%fault, we present an automatic correction technique. Second, we propose the systematic approach that can automatically correct all or part of the misclassi?ed packets
%of a faulty ?rewall policy. To the best of our knowledge,
%our paper is the ?rst one for automatic correction of ?rewall policy faults. Last, we implemented our approach
%and evaluated its effectiveness on real-life ?rewalls. To
%measure the effectiveness of our approach, we propose
%two metrics, which we believe are general metrics for
%measuring the effectiveness of ?rewall policy correction
%tools. The experimental results demonstrated that our approach is effective to correct a faulty ?rewall policy with
%three types of faults:
%
%Conclusion here!!