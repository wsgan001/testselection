\section{Approach}
\label{sec:approach}

%Figure~\ref{fig:approach} shows the overview of our approach. 

As manually selecting system tests for regression testing is tedious and error-prone, we have developed three techniques to automatically
select system tests for regression in policy evolution. Among existing system tests, the objective is to select
all system tests for regressing testing as follows.
Our approach takes two versions of program code, which interact with $v_1$ (original) and $v_2$ (new) access control policies, respectively. Our approach takes existing system tests as an input; these tests invoke methods in program code.
Our approach analyzes given program code and polices to select \emph{only} system tests for regression testing in case of policy evolution. Among given system tests, our selected system tests invoke methods to reveal changed policy behaviors between $v_1$ and $v_2$.

While we select all of system tests for regression, our test selection techniques cannot guarantee sufficiency of regression testing (i.e., whether  tests are sufficient to reveal all of changed policy behaviors). We have developed a technique to automatically augment new system
tests to satisfy sufficiency as follows. We measure sufficiency of regressing testing with our selected system tests based on rule coverage criteria~\cite{}. For not-covered changed policy behaviors (i.e., rules), our approach automatically generates system tests to cover such behaviors.

Formally, $C_1$ denotes program code, which interacts an access control policy $P_1$. $C_2$ denotes program code, which interacts a modified access control policy $P_2$. $T_0$ denotes an initial test suite for $C_1$. Our first step involves the regression test selection. We select $T'$ $\subseteq$ $T_0$ where $T'$ is a set of test cases. $T'$ execute on $C_2$ and reveal changed policy behaviors between $P_1$ and $P_2$. We measure coverage of changed policy behaviors of $P_1$ and $P_2$ with $T'$. If we find not-covered policy behaviors, we create $T''$, a set of new system tests for $C_2$. 

\Comment{
Briefly, our approach works as follows.

\begin{itemize}
	\item Policy Change Impact Analysis: Our approach conducts change impact analysis on two versions $v_1$ and $v_2$ of an access control policy.
Our approach records changed policy behaviors such as which request sets can be evaluated to different decisions for $v_1$ and $v_2$, respectively.
For a changed
	\item Test Selection: In this step, our approach executes existing test cases to record which requests are generated and evaluated against an access control policy through PDP. If a request is a subset in request sets, which reveal changed policy behaviors, our
approach selects corresponding test cases. In this paper, we develop three different test selection
techniques; the first technique is , the second one is, and the third one is	
	\item Coverage Measurement: to measure sufficiency of our selected test cases in terms of revealing changed
	policy behaviors, our approach measure policy coverage based on changed behaviors. We minimize test minimization in terms of regression
	coverage, and record not-covered rules in terms of changed behaviors.
	\item Test Augmentation: in this step, we generate test cases to cover not-covered rules. In order to create test cases
	in practice, we first find request sets and existing test case with high similarity using test code behaviors using symbolic
	execution. Then, our approach recommends existing test case candidates for augmentation. We then, manually modify test
	cases to be amendable for such request sets.
\end{itemize}			
}


We next describe our proposed three test selection techniques and test augmentation technique.


\subsection{Test Selection via Mutation Analysis}

%[\textbf{Two problems}- preselection, so if new rule or deletion, cannot hardly track. other mutants.. okay?]
%[preselectionn techniques for tests, for new test cases, we only evalue test evaluations.]

Our first proposed technique uses mutation analysis to select system tests as follows.
First step is to set up rule-test correlation.
Given a policy $P$, we creates its rule-decision-change (RDC) mutant policy $Pr_i$ by changing decision (e.g., Permit to Deny) of rule r$_i$ in turn. An example mutant policy is shown in Figure~\ref{fig:rdcexample}. The mutant policy mutates Rule 1's decision in the original example policy in Figure~\ref{fig:example}; original Rule 1's decision Permit is changed to Deny. The technique finds affected tests for this rule decision change. 
The technique executes system tests $T$ on program code for $P$ and $Pr_i$, respectively. To detect changed policy behaviors, the technique monitors responses of evaluated requests formulated from system tests $ST$. When the technique find system tests, which evaluate different policy decisions against $P$ and $P'$, respectively, our technique 
correlates rule r$_i$ and $ST$.
We continue our technique until we find each rule's correlated system test cases in turn.

Second step is to select system tests for regression on $P$ and its modified policy $P_m$.
Our technique conducts change impact analysis of $P$ and $P_m$ to find which rules' decision are changed.
We select system tests correlated with such decision changed rules.

While the technique can quickly select system tests in the second step, the technique requires rule-test setup (in the first step), which could be costly in terms of execution times. Given n rules in $P$, we execute $T$ for 2$\times$n times. 
As the first step is applied for only existing rules $R$ in $P$, our technique requires addition of rule-test
correlation for newly added rules $R_n$ where $R_n$ $\notin$ $R$ in $P_m$. 
In addition, if a new system test is introduced, we execute this test for 2$\times$n times.
However, an important benefit is that we are enabled to conduct rule-test set-up once before encountering policy changes in terms of correlated rules. 

\begin{figure}[t]%{t}
\begin{CodeOut}
\begin{alltt}
 1 <Policy PolicyId="\textbf{Library Policy}" RuleCombAlgId="\textbf{Permit-overrides}">
 2  <Target/>
 3    <Rule RuleId="\textbf{1}" Effect="\textbf{Deny}">
 4      <Target>
 5        <Subjects><Subject> \textbf{BORROWER} </Subject></Subjects>
 6        <Resources><Resource> \textbf{BOOK} </Resource></Resources>
 7        <Actions><Action> \textbf{BORROWERACTIVITY} </Action></Actions>
 8      </Target>
 9	    <Condition>
10        <AttributeValue> \textbf{WORKINGDAYS} </AttributeValue>
11      </Condition>
12    </Rule>
...
35 </policy>
\end{alltt}
\end{CodeOut}
\vspace*{-3.0ex} \caption{An example mutant policy by changing $R1$'s rule decision (i.e., effect)}
 \label{fig:rdcexample}
\end{figure}

\begin{algorithmic}
\begin{algorithm}[t]
\caption{Test Selection via Mutation Analysis Algorithm}
\STATE{\textit{TestSelection1}($P$, $P_{m}$, $T$): $T'$}
\STATE \textbf{Input:} XACML Policy $P$, modified policy $P_{m}$, Initial Sytem Test Cases $T$
\STATE \textbf{Output:} $T' \subseteq T$ where $T'$ is the subset of $T$ selected for use in regression testing $P_{m}$
\STATE {$T'$=\O{}}
\STATE /*Rule-test set-up phase*/
\FOR {each rule $r_{i}$ in Policy $P$}
\STATE {$T_{r_{i}}$=\O{}} where $T_{r_{i}} \subseteq T$ are the tests correlated to $r_{i}$ 
\STATE {/*We mutate the policy $P$ by creating a rule-decision change (RDC) on $r_{i}$ to get $P_{r_{i}}$*/}
\STATE {$P_{r_{i}} \xleftarrow[RDC(r_{i})]{} P$}
\STATE {Execute $T$ with $P_{r_{i}}$}
\FOR {each $t$ in $T$}
\STATE Let $E(t)$ be the test execution result, $E(t)={Success, Failure}$
\IF{$E(t) \leftarrow$ Failure} 
\STATE $T_{r_{i}} \leftarrow T_{r_{i}} \cup t$
\ENDIF
\ENDFOR
\STATE Map($r_{i}$,$T_{r_{i}}$)
\ENDFOR
\STATE /*Test selection phase*/
\STATE $\{r_{m}\}_{i=1..m} \leftarrow diff(P,P_{m})$
\FOR {Each rule $r_{i}$ in $\{r_{m}\}_{i=1..m}$}
\STATE {$T' \leftarrow T' \cup T_{r_{i}}$}
\ENDFOR
\STATE return $T'$
\end{algorithm}
\end{algorithmic}

\subsection{Test Selection via System Test Execution}

Our previous technique finds correlation of all of existing rules $N$ in a given policy. To reduce
such correlation setup efforts, we develop a technique to correlate only rules, which can be evaluated
by system tests. Our intuition is that system tests may interact only small number of rules in a policy
instead of a whole rules in the policy. Therefore, we require correlation of system tests for only such small number of rules.
First step is to set up rule-test correlation.
The technique executes system tests $T$ on program code for $P$. The technique monitors which rules in a policy are evaluated with
requests formulated from system tests $ST$. Our technique correlates rule r$_i$ and $ST$.
We continue our technique until we find correlated system test cases with rules in a policy.
Second step is the same with the previous technique. This step is to select system tests for regression on $P$ and its modified policy $P_m$.
Our technique conducts change impact analysis of $P$ and $P_m$ to find which rules' decision are changed.
We select system tests correlated with such decision changed rules.
Both the two techniques are white-box testing, where access control policies are available.
An important benefit of this technique is to reduce cost in terms of mutation analysis and execution times. This technique does not require generating mutants by changing rule's decision in turn. Moreover, the technique can significantly reduce execution time.
While the technique can quickly select system selects in the second step, the technique requires rule-test setup (in the first step), which could be costly in terms of execution times. Consider that requests $rs$ are formulated from system tests interact only n$_1$ rules (n$_1$ $\leq$ n) in a policy.
We execute $T$ only once. Our technique requires addition of rule-test
correlation for newly added rules $R_n$ where $R_n$ $\notin$ $R$ in $P_m$ as the same with the previous technique.

\begin{algorithmic}
\begin{algorithm}[t]
\caption{Test Selection via System Test Execution}
\STATE{\textit{TestSelection2}($P$, $P_{m}$, $T$): $T'$}
\STATE \textbf{Input:} XACML Policy $P$, modified policy $P_{m}$, Initial Sytem Test Cases $T$
\STATE \textbf{Output:} $T' \subseteq T$ where $T'$ is the subset of $T$ selected for use in regression testing $P_{m}$
\STATE {$T'$=\O{}}
\STATE /*Rule-test set-up phase*/
\FOR {Each test case $t$ in $T$}
\STATE {Execute $t$ with $P$}
\STATE $MAP_{1}$=Map($t$,$\{r_{p}\}_{i=1..p}$) where $\{r_{p}\}_{i=1..p}$ are the rules in $P$ that are triggered by $t$
\ENDFOR
\STATE Generate a mapping rule/test from $MAP_{1}$
\STATE $MAP_{2}$=Map($r_{i}$,$T_{r_{i}}) \leftarrow MAP_{1}$
\STATE /*Test selection phase*/
\STATE $\{r_{m}\}_{i=1..m} \leftarrow diff(P,P_{m})$
\FOR {Each rule $r_{i}$ in $\{r_{m}\}_{i=1..m}$}
\STATE {$T' \leftarrow T' \cup T_{r_{i}}$}
\ENDFOR
\STATE return $T'$
\end{algorithm}
\end{algorithmic}

\subsection{Test Selection via Play Back}

To reduce such correlation setup efforts in the previous techniques, we develop
a technique, which does not require correlation setup. 
Our approach executes system tests $T$ on program code for $P$. Our approach records all requests issued to policy decision point (PDP) for each 
system test case. 
For test selection, our technique evaluates all issued requests against $P$ and $P_m$, respectively.
Our technique selects requests (with corresponding system test cases) to evaluate different decisions for two different policy versions 

Our previous two techniques require correlation rule-test setup. The techniques analyze two versions of policies statically or dynamically 
under test to find which rules' are changed. However, these technique require correlation setup.
For this approach, we execute all of system test cases for only once. For additional system tests, we
execute the tests only once. An important benefit is that, we don't require that access control policies are available as the preceding two
approaches. For security reasons, developers may not want to reveal their access control policies.

\begin{algorithmic}
\begin{algorithm}[t]
\caption{Test Selection via Play Back}
\STATE{\textit{TestSelection3}($P$, $P_{m}$, $T$): $T'$}
\STATE \textbf{Input:} XACML Policy $P$, modified policy $P_{m}$, Initial Sytem Test Cases $T$
\STATE \textbf{Output:} $T' \subseteq T$ where $T'$ is the subset of $T$ selected for use in regression testing $P_{m}$
\STATE {$T'$=\O{}}
\STATE {$R_{T'}$=\O{}} where $R_{T'}$ are the requests corresponding to $T'$ 
\STATE {Execute system requests $R$}
\FOR {each request $Re$ in $R$}
\IF {$decision(Re/_{P}) \neq decision(Re/_{P_{m})}$}
\STATE {$R_{T'} \leftarrow R_{T'} \cup Re$}
\ENDIF
\ENDFOR
\STATE {$T' \leftarrow R_{T'}$}
\STATE return $T'$
\end{algorithm}
\end{algorithmic}


\subsection{Test Augmentation}
TBD








