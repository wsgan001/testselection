\section{Approach}
\label{sec:approach}
As manually selecting system tests for regression testing is tedious and error-prone, we have developed three techniques to automate 
system tests selection for regression in policy evolution. Among existing system tests, the objective is to select 
all system tests for regressing testing as follows.

Our approach takes two versions of program code, which interact with $v_1$ (original) and $v_2$ (new) access control policy, 
respectively. The existing system tests are taken an input; these tests invoke methods in program code.
We analyze given program code and polices to select \emph{only} system tests for regression testing in case of policy evolution. 
Among given system tests, our selected system tests invoke methods to reveal changed policy behaviors between $v_1$ and $v_2$.

Our test selection techniques cannot guarantee sufficiency of regression 
testing (i.e., whether these tests are sufficient to reveal all of changed policy behaviors). We have developed a technique to 
automatically augment new system tests to satisfy sufficiency as follows. We measure sufficiency of regressing testing with our 
selected system tests based on rule coverage criteria~\cite{}. For not-covered changed policy behaviors (i.e., rules), 
our approach automatically generates system tests to cover such behaviors.

Formally, $C$ denotes a program code, which interacts with an access control policy $P$. $P_{m}$ is the modified version of $P$. 
$T$ denotes an initial test suite for $C$. Our first step involves the regression test 
selection. We select $T'$ $\subseteq$ $T$ where $T'$ is a set of test cases. $T'$ execute on $C$ and reveal changed policy 
behaviors between $P$ and $P_{m}$. In the second step, we measure coverage of changed policy behaviors of $P$ and $P_{m}$ with $T'$. 
If we find not-covered policy behaviors, we augment $T'$ and create $T''$ to cover all changed rules. 

\Comment{
Briefly, our approach works as follows.

\begin{itemize}
	\item Policy Change Impact Analysis: Our approach conducts change impact analysis on two versions $v_1$ and $v_2$ of an access control policy.
Our approach records changed policy behaviors such as which request sets can be evaluated to different decisions for $v_1$ and $v_2$, respectively.
For a changed
	\item Test Selection: In this step, our approach executes existing test cases to record which requests are generated and evaluated against an access control policy through PDP. If a request is a subset in request sets, which reveal changed policy behaviors, our
approach selects corresponding test cases. In this paper, we develop three different test selection
techniques; the first technique is , the second one is, and the third one is	
	\item Coverage Measurement: to measure sufficiency of our selected test cases in terms of revealing changed
	policy behaviors, our approach measure policy coverage based on changed behaviors. We minimize test minimization in terms of regression
	coverage, and record not-covered rules in terms of changed behaviors.
	\item Test Augmentation: in this step, we generate test cases to cover not-covered rules. In order to create test cases
	in practice, we first find request sets and existing test case with high similarity using test code behaviors using symbolic
	execution. Then, our approach recommends existing test case candidates for augmentation. We then, manually modify test
	cases to be amendable for such request sets.
\end{itemize}			
}
We next describe our proposed three test selection techniques and test augmentation technique.
\subsection{Test Selection via Mutation Analysis}
Our first proposed test selection technique uses mutation analysis to select system tests as follows. 
The approach needs a preliminary step which is necessary to establish a rule-test correlation. 
Given a policy $P$, we create its rule-decision-change (RDC) mutant policy $Pr_i$ by changing decision (e.g., Permit to Deny) 
of one single rule $r_i$ in $P$. An example of a mutated policy is shown in Figure~\ref{fig:rdcexample}. In this policy, 
original Rule 1's decision Permit is changed to Deny. The technique finds affected tests for this rule decision change. 
We execute system tests $T$ on program code for $P$ and $Pr_i$. To detect changed policy behaviors, 
the technique monitors responses of evaluated requests formulated from system tests $T$. The system tests, which evaluate different 
policy decisions against $P$ and $P'$, enable to map rule $r_i$ to system Tests $t \in T$. The preliminary step ends by establishing 
a correlation between each rule in $P$ and corresponding tests $t \in T$ that trigger this rule.

The selection of system tests for regression on $P$ and its modified policy $P_m$ starts by conducting change impact analysis 
of $P$ and $P_m$ to find which rules' decision are changed. 
Once these rules are identified, we use the mapping established in the preliminary phase to select the subset of 
system tests which are correlated with changed rules.

While the technique can quickly select system tests, the technique requires rule-test setup 
(in the preliminary step), which could be costly in terms of execution times. Given n rules in $P$, we execute $T$ for 2$\times$n times. 
As the preliminary step is applied for only existing rules $R$ in $P$, our technique requires addition of rule-test
correlation for newly added rules $R_n$ where $R_n$ $\notin$ $R$ in $P_m$. 
In addition, if a new system test is introduced, we execute this test for 2$\times$n times.
However, an important benefit is that we are enabled to conduct rule-test set-up once before encountering policy 
changes in terms of correlated rules. 

\begin{figure}[t]%{t}
\begin{CodeOut}
\begin{alltt}
 1 <Policy PolicyId="\textbf{Library Policy}" RuleCombAlgId="\textbf{Permit-overrides}">
 2  <Target/>
 3    <Rule RuleId="\textbf{1}" Effect="\textbf{Deny}">
 4      <Target>
 5        <Subjects><Subject> \textbf{BORROWER} </Subject></Subjects>
 6        <Resources><Resource> \textbf{BOOK} </Resource></Resources>
 7        <Actions><Action> \textbf{BORROWERACTIVITY} </Action></Actions>
 8      </Target>
 9	    <Condition>
10        <AttributeValue> \textbf{WORKINGDAYS} </AttributeValue>
11      </Condition>
12    </Rule>
...
35 </policy>
\end{alltt}
\end{CodeOut}
\vspace*{-3.0ex} \caption{An example mutant policy by changing $R1$'s rule decision (i.e., effect)}
 \label{fig:rdcexample}
\end{figure}

\begin{algorithmic}
\begin{algorithm}[t]
\caption{Test Selection via Mutation Analysis Algorithm}
\STATE{\textit{TestSelection1}($P$, $P_{m}$, $T$): $T'$}
\STATE \textbf{Input:} XACML Policy $P$, modified policy $P_{m}$, Initial Sytem Test Cases $T$
\STATE \textbf{Output:} $T' \subseteq T$ where $T'$ is the subset of $T$ selected for use in regression testing $P_{m}$
\STATE {$T'$=\O{}}
\STATE /*Rule-test set-up phase*/
\FOR {each rule $r_{i}$ in Policy $P$}
\STATE {$T_{r_{i}}$=\O{}} where $T_{r_{i}} \subseteq T$ are the tests correlated to $r_{i}$ 
\STATE {/*We mutate the policy $P$ by creating a rule-decision change (RDC) on $r_{i}$ to get $P_{r_{i}}$*/}
\STATE {$P_{r_{i}} \xleftarrow[RDC(r_{i})]{} P$}
\STATE {Execute $T$ with $P_{r_{i}}$}
\FOR {each $t$ in $T$}
\STATE Let $E(t)$ be the test execution result, $E(t)={Success, Failure}$
\IF{$E(t) \leftarrow$ Failure} 
\STATE $T_{r_{i}} \leftarrow T_{r_{i}} \cup t$
\ENDIF
\ENDFOR
\STATE Map($r_{i}$,$T_{r_{i}}$)
\ENDFOR
\STATE /*Test selection phase*/
\STATE $\{r_{m}\}_{i=1..m} \leftarrow diff(P,P_{m})$
\FOR {Each rule $r_{i}$ in $\{r_{m}\}_{i=1..m}$}
\STATE {$T' \leftarrow T' \cup T_{r_{i}}$}
\ENDFOR
\STATE return $T'$
\end{algorithm}
\end{algorithmic}

\subsection{Test Selection via System Test Execution}
Our previous technique finds correlation of all of existing rules $N$ in a given policy with the system Tests. To reduce
such correlation setup efforts, we develop a technique to correlate only rules, which can be evaluated
by system tests. Our intuition is that system tests may interact only with a small number of rules in a policy
instead of a whole rules in the policy. Therefore, we start by establishing a correlation between system tests and triggered rules.
Thus we execute system tests $T$ on program code that interacts with $P$. Our technique monitors which rules in a policy are evaluated with
requests formulated from system tests $T$. Once the mapping test-rule is established, we proceed like the first approach by 
conducting change impact analysis of $P$ and $P_m$ to find which rules' decision are changed. Mapped tests to those rules constitutes
 the subset of regression test selection. 


An important benefit of this technique is to reduce cost in terms of mutation analysis and execution times. This technique does not 
require generating mutants by changing rule's decision in turn. Moreover, the technique can significantly reduce execution time.
While the technique can quickly select system selects in the second step, the technique requires rule-test setup (in the preliminary step), 
which could be costly in terms of execution time. Consider that requests $Rs$ are formulated from system tests interact 
only n$_1$ rules (n$_1$ $\leq$ n) in a policy.
We execute $T$ only once. Our technique requires addition of rule-test
correlation for newly added rules $R_n$ where $R_n$ $\notin$ $R$ in $P_m$ as the same with the previous technique.

\begin{algorithmic}
\begin{algorithm}[t]
\caption{Test Selection via System Test Execution}
\STATE{\textit{TestSelection2}($P$, $P_{m}$, $T$): $T'$}
\STATE \textbf{Input:} XACML Policy $P$, modified policy $P_{m}$, Initial Sytem Test Cases $T$
\STATE \textbf{Output:} $T' \subseteq T$ where $T'$ is the subset of $T$ selected for use in regression testing $P_{m}$
\STATE {$T'$=\O{}}
\STATE /*Rule-test set-up phase*/
\FOR {Each test case $t$ in $T$}
\STATE {Execute $t$ with $P$}
\STATE $MAP_{1}$=Map($t$,$\{r_{p}\}_{i=1..p}$) where $\{r_{p}\}_{i=1..p}$ are the rules in $P$ that are triggered by $t$
\ENDFOR
\STATE Generate a mapping rule/test from $MAP_{1}$
\STATE $MAP_{2}$=Map($r_{i}$,$T_{r_{i}}) \leftarrow MAP_{1}$
\STATE /*Test selection phase*/
\STATE $\{r_{m}\}_{i=1..m} \leftarrow diff(P,P_{m})$
\FOR {Each rule $r_{i}$ in $\{r_{m}\}_{i=1..m}$}
\STATE {$T' \leftarrow T' \cup T_{r_{i}}$}
\ENDFOR
\STATE return $T'$
\end{algorithm}
\end{algorithmic}

\subsection{Test Selection via Play Back}
To reduce such correlation setup efforts in the previous techniques, we develop
a technique, which does not require a correlation setup. 
Our approach executes system tests $T$ on program code for $P$ and records all requests issued to policy decision point (PDP) for each 
system test case. For test selection, our technique evaluates all issued requests against $P$ and $P_m$ and selects the test subset of
 requests (with corresponding system test cases) that engender different decisions for two different policy versions.

While our previous two techniques require correlation rule-test setup. 
%The techniques analyze two versions of policies statically 
%or dynamically under test to find which rules' are changed. However, these technique require correlation setup.
The current approach requires the execution of system test cases only once.
% For additional system tests, we execute the tests only once. 
Moreover, while the two previous techniques are white-box testing since access control policies are available, the present technique 
does not require the availability of access control policies. This can present a considerable advantage when developers don't want 
to reveal their access control policies.

\begin{algorithmic}
\begin{algorithm}[t]
\caption{Test Selection via Play Back}
\STATE{\textit{TestSelection3}($P$, $P_{m}$, $T$): $T'$}
\STATE \textbf{Input:} XACML Policy $P$, modified policy $P_{m}$, Initial Sytem Test Cases $T$
\STATE \textbf{Output:} $T' \subseteq T$ where $T'$ is the subset of $T$ selected for use in regression testing $P_{m}$
\STATE {$T'$=\O{}}
\STATE {$R_{T'}$=\O{}} where $R_{T'}$ are the requests corresponding to $T'$ 
\STATE {Execute system requests $R$}
\FOR {each request $Re$ in $R$}
\IF {$decision(Re/_{P}) \neq decision(Re/_{P_{m})}$}
\STATE {$R_{T'} \leftarrow R_{T'} \cup Re$}
\ENDIF
\ENDFOR
\STATE {$T' \leftarrow R_{T'}$}
\STATE return $T'$
\end{algorithm}
\end{algorithmic}


\subsection{Test Augmentation}
TBD








