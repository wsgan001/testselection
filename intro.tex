\section{Introduction} \label{sec:introduction}
 
Access control is one of the privacy and security mechanisms for granting only legitimate users with access to critical information. 
Access control is governed by access control policies (policies in short), each of which includes a sequence of rules to specify 
which subjects are permitted or denied to access which resources under which conditions. To facilitate specifying and maintaining policies separately from entangled code, 
system developers specify and enforce policies separated from actual functionality (i.e., business logic) of a system.

In the system, system behaviors are dependent on not only program code but also policies, which
are available in various policy specification formats such as XACML.
Program code, which represents the actual functionality,
interacts with a policy through a security component, called Policy Decision Point (PDP).
Consider that the program code consists of methods $Ms$ including Policy Enforcement Points (PEPs).
PEPs are functionalities that require decisions (e.g., permit or deny) whether a given subject can have access on protected information.
Typically, PEPs in $Ms$ formulate an access request that specifies that a subject would like to have a specific type of access (e.g., read or write) on specific protected information. 
The PEPs next submit the request to a PDP, which evaluates the request against the policy loaded on the PDP and determines whether the request should be permitted or denied. Finally, 
the PDP formulates and sends the
decision back to the PEPs to proceed.

%Or, a database trigger could be modified, which would affect the behavior of all those test cases whose database interactions cause the modified trigger to be executed.


As security requirements of software often change during software development and maintenance,
developers may modify policies to comply with requirements.
%For example, rules in policies may be added, deleted, or modified.
In consequence, changes in policies can affect system's behavior in various ways such as
changes of access control privileges of a group or users that the system supports for.
In order to increase confidence that the modification of policies is correct and
does not introduce unexpected behavior, developers periodically conduct regression testing.
For example, new security requirements include new security concerns to be added into a policy.
Developers may change policies without changing program code related to actual system functionality.
In such a situation, validating and verifying program code and policies together after policy changes
increases confidence of correct behaviors of the given system.


In this paper, we focus on the regression testing problem in the context of policy evolution.
For policy evolution, regression testing is important because policy behavior changes may
result in unexpected behaviors of program code, these behaviors can be undesirable.
Prior research work on test selection~\cite{Rothermel:1996:ART:235681.235682,Graves:2001:ESR:367008.367020,Elbaum:2000:PTC:347324.348910}
deals with changes in program code.
However, this prior work does not consider code-related components such as policies that interact with the program code.
To the best of our knowledge,
our paper is the first one for automatic test selection in the context of policy
evolution.

\FixJeeHyun{[TBD-JeeHyun: add examples to illustrate would help!!]}
The typical regression testing for program code interacting with a policy is as follows.
Given a program and a policy $P$, the developers prepare initial system test cases, where
each test case maps to rules (in the policy) exercised by the test case. Given $P$ and its modified
policy $P'$, the developers compare impacts of $P$ and $P'$ to
reveal different policy behaviors, which are ``dangerous'' portions to be validated with
test cases. For validating these ``dangerous'' portions, the developers often select the test cases (from test cases for $P$) that exercise the dangerous
portions of $P'$.

For regression testing, instead of writing new system test cases, developers reuse the initial test cases in practice. A naive regression testing strategy is to rerun all system test cases. However,
 this strategy is costly and time-consuming, especially for large-scale systems. Moreover, if the number of the initial 
system test cases is large, this strategy may require significant time for developers to conduct testing. In order to
 reduce the cost of regression testing, developers often adopt regression test selection, which selects and executes only
 test cases to expose different behaviors across different versions of the system. This approach sometimes may require substantial
 costs to select and execute such system test cases from the initial test cases that could reveal faults introduced by the changes. 
If the cost of regression test selection is smaller than rerunning all of initial system test cases, test selection helps reduce overall cost in validating whether the modification is correct.

\review{Besides of reducing overall cost, safeness is important as well. Safe regression-test-selection approach
selects every test case that may reveal a fault in modified program~\cite{Rothermel:1996:ART:235681.23568}.
In contrast, unsafe regression-test-selection approach may omit test cases
that may reveal a fault in the modified program. Similarly, we use
safeness in this paper in case of policy evolution by referring modified program
as modified policies.}

To address these issues, we propose safe regression-test-selection approach.
Given an original 
policy $P$ and its modified policy $P'$,
our approach selects superset $SU_t$ of the set of test cases (i.e., fault-revealing test cases) that reveal
faults for $P'$ by analyzing different behaviors in program code due to policy changes.
%$SU_t$ includes not only fault-revealing test cases and but also
%modified-policy-behavior-revealing test cases (i.e., test cases that do not reveal faults but only reveal different policy behaviors).
%Intuitively, requests triggered from $SU_t$ may
%reveal different policy behaviors between $P$ and $P'$.
In the policy context, different policy 
behaviors refer to that, given a request, its evaluated decisions (for $P$ and $P'$, respectively) are different.
In our notion, we refer rules $R_imp$ impacted by policy changes as rules showing such different policy behaviors due to policy changes.
These different policy behaviors are dangerous portions to be validated.


%Our test-selection approach first analyzes traceability that links test cases
%with rules impacted by policy changes.

Our test-selection approach includes three techniques.
The first one uses mutation analysis.
This technique first analyzes correlation between test cases and rules impacted by policy changes.
%$R_imp$ denote rules that requests applicable $R_imp$ reveal different decisions (for $P$ and $P'$, respectively). 
For correlation setup, this technique creates a mutant $P'(r_i)$
by changing decision of rule $r_i$ in $P$. On executing
test cases on program code interacting with $P$ and $P'(r_i)$,
respectively, this technique collects test cases that relate to $r_i$
by analyzing different policy behaviors induced by policy changes.
While this technique is easy to implement, this technique is costly
in terms of test case execution.
Given the number $n$ of rules in $P$, this technique requires at least
$n$ execution of test cases against mutants to find all correlation between
test cases and rules. This technique next conducts change-impact-analysis statically that analyzes changes of rules between $P$ and $P'$
to select regression-test-cases.
 
To reduce cost in terms of test case execution, the second one uses coverage analysis
for correlation setup.
While executing tests cases against $P$, this technique finds correlation between
test cases and rules by analyzing which rules are covered (i.e., evaluated) with each test
case. This technique requires test case execution at once.
However, these first two techniques use change-impact-analysis which is often
costly in terms of time by comparing all policy behaviors between $P$ and $P'$ statically.

To reduce such change-impact-analysis cost,
the third one first captures requests triggered from test cases at runtime.
This technique next evaluates these requests against $P$ and $P'$ and
select only requests (that to reveal different
policy behaviors) and corresponding test cases. 

 

Our main objective is to, through test-selection, is to detect faults efficiently in the modified
policy.
While test-selection techniques are useful to reveal faults efficiently caused by policy changes, these test cases may not 
\FixJeeHyun{cover all the rules $R_{imp}$ that are impacted by those policy changes}.
In other words, the selected test cases may fail to cover a rule including a fault in the modified policy.
%Policy testers may generate and select a test suite to
%achieve a certain (high) level of coverage. However, our
%main objective, through testing, is to detect faults in the
%?rewall policy while reaching a certain level of coverage.
%\FixJeeHyun{What is consequence of...}
To address this issue, we propose a test-augmentation technique, which complements our test-selection techniques.
\FixJeeHyun{Note that we first use a regression-test-selection technique before using the test-augmentation technique that generates the additional test cases 
that cover rules $\in$ $R_{imp}$ that are not covered by existing test cases.}
Covering those rules helps investigate a larger portion of changed policy behaviors for fault detection using additional test cases that achieve higher rule coverage.


\Comment{
While test-selection techniques are useful for selecting test cases for program code impacted by policy changes, these test cases may not sufficiently cover all the rules in the policy impacted by the changes.
To address this issue, we propose a test-augmentation technique, which complements our test-selection techniques.
Note that we first use a regression-test-selection technique before using the test-augmentation technique to achieve
additional policy coverage for not-covered but impacted rules $NR$ by existing test cases.
Our test-augmentation technique analyzes and recommends existing test cases, which can issue requests $NR_{req}$ cover $NR$
after manual modification of the test cases.
In the technique, we use prioritization to classify test cases into various sets based on their modification likelihood
to issue $NR_{req}$.
High likelihood may lead to less number of test case modification to issue $NR_{req}$.
Given test cases with high likelihood, we manually modify and augment the test cases to increase coverage of $NR$.
}

%For augmented test cases,
%we then verify that augmented test cases increase coverage of $N_r$.
%For example, an attribute change (e.g., subject role change from $Student$ to $Professor$) may cover $N_r$.



%structural test generation before using combinatorial test generation to augment the generated test inputs for achieving the
%n-wise (e.g., pairwise) coverage criterion. The reason for
%using combinatorial test generation after structural test generation is that structural test generation (white-box testing)
%cannot detect some types of faults including omission faults
%(e.g., missing a rule).



This paper makes three main contributions:

\begin{itemize}
  \item We develop a test selection approach to select every test case from existing test cases to reveal different policy behaviors due to policy changes. Our approach
  uses three techniques; the first one is based on mutation analysis, the second one is based on coverage analysis, and the third one is based on evaluated 
decisions of requests issued from test cases. 
  \item We develop a test augmentation technique to generate additional new test cases to cover not-covered rules among rules impacted by policy changes with selected test cases by the preceding techniques.

  \item We evaluate our approach on three real world Java programs interacting with policies. Our evaluation results show that our test selection techniques achieve
 up to 51\%$\sim$97\% of test reduction for a modified version with given 5$\sim$25 policy changes. Among our three techniques, our results show that the 
third technique is the most efficient compared with the first
and the second techniques in terms of elapsed time. The third technique is 43 and 8 times
faster than the first and the second techniques, respectively.
Moreover, our evaluation results show that our selected test cases detect XX regression faults (on average) caused by policy changes.
To increase confidence on correctness of the modified version, our test augmentation technique generates additional test cases to cover additional 26\% of the not-covered rules impacted by policy changes.
\end{itemize}

The rest of the paper is organized as follows.
Section~\ref{sec:background} presents background information about
policy-based software systems, policy context, and regression testing.
Section~\ref{sec:approach} presents our approach.
Section~\ref{sec:implementation} presents our implementation. 
Section~\ref{sec:experiment} describes the evaluation results
where we apply our approach on three projects.
Section~\ref{sec:discussion} discusses issues. 
Section~\ref{sec:related} discusses related
work. Section~\ref{sec:conclusion}
concludes the paper.






%
%[oracles]
